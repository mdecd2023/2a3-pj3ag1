\renewcommand{\baselinestretch}{1.5} %設定行距
\pagenumbering{roman} %設定頁數為羅馬數字
\clearpage  %設定頁數開始編譯
\sectionef
\addcontentsline{toc}{chapter}{摘~~~要} %將摘要加入目錄
\begin{center}
\LARGE\textbf{摘~~要}\\
\end{center}
\begin{flushleft}
\fontsize{14pt}{20pt}\sectionef\hspace{12pt}\quad BubbleRob範例是CoppeliaSim中的一個機器人控制示例。該範例中,機器人由兩個輪子驅動,可以在環境中移動並與氣泡互動。在模擬開始時,機器人和氣泡的初始位置被設置,並根據感測器信息和控制策略來控制機器人的移動。使用接近傳感器檢測機器人前方的氣泡,根據感測器信息,機器人可以調整輪子的速度以控制移動方向和速度。當機器人接觸到氣泡時,可以觸發互動操作。整個模擬過程中可以設置結束條件,例如運行時間或達到特定目標。BubbleRob範例通過展示機器人的移動和互動,演示了基於感測器信息和控制策略的機器人控制過程。\\[12pt]

\end{flushleft}
\begin{center}
\fontsize{14pt}{20pt}\selectfont 關鍵字:  \sectionef CoppeliaSim、bubbleRob
\end{center}
\newpage
%=--------------------Abstract----------------------=%
\renewcommand{\baselinestretch}{1.5} %設定行距
\addcontentsline{toc}{chapter}{Abstract} %將摘要加入目錄
\begin{center}
\LARGE\textbf\sectionef{Abstract}\\
\begin{flushleft}
\fontsize{14pt}{16pt}\sectionef\hspace{12pt}\quad The BubbleRob example is a robot control demonstration in CoppeliaSim. In this example, the robot is driven by two wheels and can move in the environment and interact with bubbles. At the start of the simulation, the initial positions of the robot and bubbles are set, and the robot's movement is controlled based on sensor information and control strategies. Using proximity sensors, the robot detects bubbles in front of it, and based on the sensor information, it can adjust the speed of its wheels to control the direction and velocity of movement. When the robot comes into contact with a bubble, it can trigger interaction operations. Throughout the simulation, termination conditions can be set, such as running time or reaching specific goals. The BubbleRob example showcases the robot control process based on sensor information and control strategies by demonstrating the robot's movement and interaction.\\[12pt]
\end{flushleft}
\begin{center}
\fontsize{14pt}{16pt}\selectfont\sectionef Keyword:  CoppeliaSim、bubbleRob
\end{center}
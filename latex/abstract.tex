\renewcommand{\baselinestretch}{1.5} %設定行距
\pagenumbering{roman} %設定頁數為羅馬數字
\clearpage  %設定頁數開始編譯
\sectionef
\addcontentsline{toc}{chapter}{摘~~~要} %將摘要加入目錄
\begin{center}
\LARGE\textbf{摘~~要}\\
\end{center}
\begin{flushleft}
\fontsize{14pt}{20pt}\sectionef\hspace{12pt}\quad 本課程將採兩人一組、四人一組與八人一組的方式進行協同機電整合產品開發,開發一款能在 web-based CoppeliaSim 場景中雙方或多方 (human or computer) 對玩的遊戲 (game) 產品。最後在 w16 現場發表八人協同四週後所完成的產品,在 w17 各組採 OBS + Teams 以影片發表所完成的協同產品。\\[12pt]

\end{flushleft}
\begin{center}
\fontsize{14pt}{20pt}\sectionef\hspace{12pt}\quad 課程一開始讓學員可以從 https://mde.tw/cd2023/content/BubbleRob.html 導引練習中,了解 CoppeliaSim 套件中的諸多功能以及用法,其中包括利用近接感測器偵測障礙物,並透過 Lua script 控制 bubbleRob 雙輪車的移動。為了讓各組學員了解在多人協同模式下,開發機電資產品流程中必須面臨的許多議題(若要直接在瀏覽器中建立多方協同的場景,可以透過 remote API(導引) 與 Visualization Stream 功能)。再接續專案一的雙輪車,改用 Python zmqRemoteAPI 進行控制,各分組需完成能在 Visualization Stream 瀏覽器中,跨網路雙方各控制一台雙輪車在足球場中進行對陣,且需設計一組能在雙方瀏覽器中進行計分的系統。最後各組需對雙輪車進行設計改良,以提升行進與對戰效能,各組需採 CAD 進行場景與多輪車零組件設計後,轉入足球場景中以鍵盤 arrow keys 與 wzas 等按鍵進行控制,對陣雙方每組將有四名輪車球員,且每兩人在同一台電腦上操作,完成後各組需在分組網站中提供所有相關檔案下載連結,且提供線上分組簡報與分組 pdf 報告連結。\\[12pt]

\end{flushleft}
\begin{center}
\fontsize{14pt}{20pt}\sectionef\hspace{12pt}\quad 專案場景必須要有感測器及記分板,讓進球後可以顯示分數在場景上,而記分板除了採用 LED 顯示計分外,也要以建立以機械轉盤傳動計分系統。另外建立計時器讓學員在對戰時得知時間剩多少,並利用程式控制球門使球重置後繼續對戰,最後在
CoppeliaSim 模擬環境中透過埠號及 http://[2001:288:6004:17:2023:cda:x:x]:23020/ 進行對戰及觀看。\\[12pt]
\newpage
%=--------------------Abstract----------------------=%
\renewcommand{\baselinestretch}{1.5} %設定行距
\addcontentsline{toc}{chapter}{Abstract} %將摘要加入目錄
\begin{center}
\LARGE\textbf\sectionef{Abstract}\\
\begin{flushleft}
\fontsize{14pt}{16pt}\sectionef\hspace{12pt}\quad This course will involve collaborative development of mechatronic integrated products in teams of two, four, and eight members. The objective is to create a web-based game product using CoppeliaSim, where two or more participants (humans or computers) can engage in gameplay within the virtual environment. At the end of the course, during week 16, the eight-member teams will present their completed products in a live demonstration. In week 17, each group will use OBS + Teams to present a video showcasing their collaborative product.\\[12pt]

\end{flushleft}
\begin{center}
\fontsize{14pt}{16pt}\sectionef\hspace{12pt}\quad At the beginning of the course, participants will be guided to practice and explore various functionalities and usage of the CoppeliaSim package through the link https://mde.tw/cd2023/content/BubbleRob.html. This includes using proximity sensors to detect obstacles and controlling the movement of the BubbleRob differential-drive robot using Lua script. In order to familiarize the teams with the challenges encountered in the development of mechatronic products in a multi-user collaborative mode, they will utilize the remote API (guided) and Visualization Stream feature to create a collaborative scenario directly in the web browser. Continuing from Project 1, the teams will switch to using Python zmqRemoteAPI for control. Each group will be required to establish a scenario where two BubbleRob robots, controlled by two separate users over the network, engage in a soccer match within the visualization stream browser. Additionally, the teams need to design a scoring system that can be accessed and displayed in the browsers of both users. In the final stage, each group will improve the design of their BubbleRob robot to enhance its movement and performance during the match. The groups will employ CAD software to design the scene and components of the robots. The control will be switched to using arrow keys and "wzas" keys on the keyboard to maneuver the robots within the soccer field. Each group will have four robot players, with each pair of participants operating on the same computer. Once completed, the groups are required to provide download links for all relevant files on the group website, along with links to online group presentations and PDF reports.\\[12pt]

\end{flushleft}
\begin{center}
\fontsize{14pt}{16pt}\sectionef\hspace{12pt}\quad 
The project scenario requires the presence of sensors and a scoreboard in the simulation environment. The scoreboard should display the score on the scene, indicating goals scored. Apart from using LED displays to show the score, a mechanical rotary-driven scoring system should also be implemented. Additionally, a timer needs to be created to inform the participants about the remaining time during the gameplay. The program should control the goal posts to reset the ball and continue the game. Finally, the participants can engage in the match and observe it through the CoppeliaSim simulation environment using the port address http://[2001:288:6004:17:2023:cda:x:x]:23020/. (Note: Please replace "x" in the provided address with the appropriate values or specific information.)\\[12pt]
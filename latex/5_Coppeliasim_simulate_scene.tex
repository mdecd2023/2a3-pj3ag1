\chapter{場景模擬}
\renewcommand{\baselinestretch}{10.0} %設定行距
\pagenumbering{arabic} %設定頁號阿拉伯數字
\setcounter{page}{18}  %設定頁數
\fontsize{14pt}{2.5pt}\sectionef
\section{摘要}
  完成球員及程式碼設定後,接著建立模擬場景,添加場地、球員、計時器、 LED 記分板和機械式轉盤記分板\\
\section{統整場景}
  在球場周圍設置感測器球碰到會返回中心,放入 8 名球員,以兩色分為兩隊。\\
\begin{figure}[hbt!]
\begin{center}
\includegraphics[height=8cm]{總2}
\caption{\Large 球場建立}\label{總2}
\end{center}
\end{figure}

  和計時器、LED計分板和機械式記分板。\\
\begin{figure}[hbt!]
\begin{center}
\includegraphics[height=15cm]{總}
\caption{\Large 球場全貌}\label{總}
\end{center}
\end{figure}
\newpage

\section{ CoppeliaSim}
  CoppeliaSim,曾被稱為V-REP(Virtual Robot Experimentation Platform),是一個功能強大且多用途的機器人仿真軟體。它允許使用者在虛擬環境中創建、模擬和分析機器人系統。CoppeliaSim提供各種功能和功能,適用於各種機器人應用、研究和教育目的。
機器人建模它提供了一個用戶友好的界面,可以設計和建模具有可自定義屬性(如形狀、大小和運動學)的複雜機器人。用戶可以從頭開始創建機器人,也可以導入現有的機器人模型。
仿真環境:CoppeliaSim提供一個3D仿真環境,用戶可以在其中模擬機器人及其與虛擬世界的交互作用。它包括物理仿真、碰撞檢測和真實動態,以模擬真實世界的情境。
傳感器仿真,CoppeliaSim提供了一個內置的腳本界面,允許使用者使用不同的語言(包括Lua、Python和MATLAB)編寫和控制機器人。這使得使用者可以實現複雜的控制算法並在仿真環境中進行測試。
多機器人系統:它支持多機器人系統的仿真,使使用者能夠設計和研究多個機器人之間的協作行為、集群機器人和協調策略。
遠程API:CoppeliaSim提供一個遠程API,允許外部應用程序與仿真進行實時通信和控制。這一功能可用於將仿真與外部硬體(如機器人控制器或人工智能算法)\\
\renewcommand{\baselinestretch}{1} %設定行距
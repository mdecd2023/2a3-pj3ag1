\chapter{場景模擬}
\renewcommand{\baselinestretch}{10.0} %設定行距
\pagenumbering{arabic} %設定頁號阿拉伯數字
\setcounter{page}{7}  %設定頁數
\fontsize{14pt}{2.5pt}\sectionef
\section{摘要}
  完成球員及程式碼設定後,接著建立模擬場景,添加場地、球員、計時器、 LED 記分板和機械式轉盤記分板\\
\section{統整場景}
  在球場周圍設置感測器球碰到會返回中心,放入 8 名球員,以兩色分為兩隊。\\
\begin{figure}[hbt!]
\begin{center}
\includegraphics[height=8cm]{60}
\caption{\Large 球場建立}\label{總2}
\end{center}
\end{figure}

  和計時器、LED計分板和機械式記分板。\\
\begin{figure}[hbt!]
\begin{center}
\includegraphics[height=15cm]{61}
\caption{\Large 球場全貌}\label{總}
\end{center}
\end{figure}
\renewcommand{\baselinestretch}{1} %設定行距
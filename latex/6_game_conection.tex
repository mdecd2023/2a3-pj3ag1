\chapter{組員連線}
\renewcommand{\baselinestretch}{10.0} %設定行距
\pagenumbering{arabic} %設定頁號阿拉伯數字
\setcounter{page}{21}  %設定頁數
\fontsize{14pt}{2.5pt}\sectionef
\section{摘要}
  完成場景建設後,要實施跨電腦連線對戰,使用zmqRemoteAPI 寫的程式和下載 CoppeliaSim(4.5.1) 支援 IPv6 版本,zmq 中也需具備 IPv6 環境,然後由組長開起場景,組員跨網路控制各自的編號球員開始對戰。\\
\section{連線說明-防火牆}
  從控制台將防火牆都關閉,點開進階設定,組長設定"輸入規則"、組員設定"輸出規則"。新增規則 / 連接埠 / TCP(傳輸控制協定Port) / 特定連接埠 23000-23050,選擇允許連線。\\
\begin{figure}[hbt!]
\begin{center}
\includegraphics[height=6cm]{防3}
\caption{\Large 控制台連接埠}\label{防3}
\end{center}
\end{figure}
\section{連線說明-IPv6}
  設定網路 IPv6 位址。\\
\begin{figure}[hbt!]
\begin{center}
\includegraphics[height=10cm]{防9}
\caption{\Large 網路 IPv6 位置}\label{防9}
\end{center}
\end{figure}

  接著在 zmq 的 localhost 處打上組長的 IPv6 位置連線。\\
\newpage
\begin{figure}[hbt!]
\begin{center}
\includegraphics[height=6cm]{連1}
\caption{\Large 組長 IPv6 }\label{連1}
\end{center}
\end{figure}
  最後在瀏覽器輸入 http://[組長 IP 位置]:23020,即可看到組長的場景。